\summary
\english

This thesis was developed while working at Barcelona Supercomputing Center, which is specialized in High Performance Computing and research in many fields, such as cloud computing, bioinformatics, material science and more.\\


The whole thesis aimed to perform the verification process on a Vector Processing Unit, taking part of the European Processor Initiative (EPI) project.
The implemented Vector Processing Unit is based on the RISC-V V-Extension specifications, and it is on develop because it aims to enhance modern computing with parallelism and efficiency.\\

The manuscript will follow all the steps useful to implement all the verification tool.
In the first chapter it is presented an introduction of all the concepts and of the context in which this thesis is been developed.\\

Then, in the second chapter, all the techniques used to verify functionally and formally this VPU are discussed.\\

Afterwards all the results, such as found bugs or created material, are displayed in the third chapter. Here all the results are used to highlight the finest approaches into the verification process. It is shown how formal and functional tools can be used to find bugs or to better define specification. \\

Finally in the conclusive chapter, it is displayed how multiple techniques are required in different situations to reduce the Verification effort, taking examples from the analysed cases.








\bigskip
