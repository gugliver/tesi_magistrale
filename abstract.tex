\summary
\english



Modern architectures are increasing in capabilities but at the same time they are getting better in low-power applications.
This  is a problem for standard architectures and it's leading to develop logic based on parallelism at high efficiency.\\

There are different of ways to take advantage of parallelism: super-scalar processors, pipelines and vectors. In particular vector extensions are used to add the power of parallelism to a simple RISC architecture. A famous example is the V-Extension of the RISC-V architecture. This allows to gain computational power and develop modularity.\\

This work is based on the EPI project, in particular the vector extension for an European Processor based on RISC-V architecture.\\








\bigskip
%This thesis is developed during an internship at the Barcelona Supercomputing Center for the EPI (European Processor Initiative) project, whose goal is to design a Vector Processing Unit based on RISC-V V-extension. The main work was about the functional and formal verification on the VPU’s submodules. The first part of the work consisted in the analysis of the general behaviour on the submodules I was in charge of, and the creation of some solid specifications together with the Design Team. Subsequently, the upgrade of the Verification Plan was done defining assertions, assumptions, tests and coverage. An UVM structure was developed to implement and test concurrent assertions to verify the functional behaviour. Afterwords those assertions were tested formally as well, using the assumptions made before. Eventually those assertions were implemented into the whole VPU structure to spot errors into longer and more complex tests.