\summary
\english

This thesis was developed while working at Barcelona Supercomputing Center, which is specialized in High Performance Computing and research in many fields, such as cloud computing, bioinformatics, material science and more.\\

The European Processor Initiative is a project for an European Open Source processor, based on RISC-V specifications. In particular the BSC is developing a Vector Processing Unit.\\
 
Modern architectures are increasing in capabilities while at the same time they need to get better in low-power applications.
This  is a problem for standard architectures and it is leading to develop logic based on parallelism at high efficiency.\\

There are multiple of ways to take advantage of parallelism: super-scalar processors, pipelines and vectors. In particular, vector extensions are used to add the power of parallelism to a simple RISC architecture. A famous example is the V-Extension of the RISC-V architecture. This module allows to gain computational power and develop modularity.\\

The work was done during the development and the verification of the Vector Processing Unit. This unit tries to solve all the needs about parallelism and modularity.\\

After an introductional chapter containing all the theoretical informations, all the techniques used to verify functionally and formally this VPU will be discussed.\\
Only afterwards all the results, such as founded bugs or created material, will be displayed in chapter 3.\\

Finally the conclusions will highlight how multiple techniques are required in different situations to reduce the Verification effort, taking examples from the analysed cases.








\bigskip
