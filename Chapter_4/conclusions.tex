\chapter{Conclusions}
\section{Conclusions}
This thesis has shown all the efforts to verify the behaviour of some DUTs, in particular for the load operations.\\

Analyzing the results obtained with the Load Management Unit, it is possible to understand why it is crucial to have a good Verification Plan. It was used to test all the specifications and in this way to test their robustness. A detailed Verification Plan enables to find some bugs in many different cases and, moreover, it was helping defining an high level model, in order to eventually create a scoreboard. 
In particular when the design is simple the usage of the scoreboard becomes very useful. In this thesis the scoreboard was created for the LMU but not for the LB, otherwise it would have required a verification process on the scoreboard itself.
Scoreboard-like assertions were then created to test the results. However, those assertions are not testing the difficult cases, so it is very hard form them to find an hidden bug. 
But it is still possible to improve them, creating a complete scoreboard. In fact, all the code created, was created following the philosophy of re-usability and modularity.
When performing some long random tests, some blind spots were still present, because the randomness needs a lot of time to reach all the corner cases. That is the reason why a Test Plan was necessary, as well as its capability to define the design behaviour in difficult cases.

The Formal tool has been used with the sole purpose of assuring the assumptions and the assertions were formulate correctly. Even though it was too early to apply different usage of this tool in this thesis, it is important to mention that there is the chance to use it for other aims, such as an optimization of the model using mathematical equivalence.


As a final consideration an estimation on the bugs found was done:
\emph{\textasciitilde 40\% of the load related problems were spotted with the work done in this thesis}. In particular, \emph{up to the 55\% of the errors on the Load Management Unit were spotted using the UVM structure, the assertions and the specification review}.
This confirms the importance of advanced techniques for the verification process. Moreover, all the tools created are still in use in the current development of the EPI project. This will allow further discoveries on bugs and missing specifications.

\section{Future developments}
Regarding the verification process, there is still some space for future developments. In particular the next steps would be to complete the Verification Plan and add assertions and assumptions for all the submodules. 
It would also be useful to have a scoreboard for each submodule and to complete the one for the Load Buffer.
Formal verification could also be used to test all the VPU and then optimize the design in order to have less logic and so less errors.
Finally, it would be more productive to implement more coverage controls. This will allow to better guide the process and to avoid blind spots.

Considering the project in its entirety, it is now important to catch up with the specifications that are being updated by RISC-V. In this way it will be possible to provide an innovative VPU.



